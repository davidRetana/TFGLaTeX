
\chapter{Apache\textsuperscript{\footnotesize\texttrademark} Hadoop\textsuperscript{\footnotesize\textregistered}}
\section{¿Qué es Apache Hadoop?}\label{sec:que_es_apache_hadoop}
\textbf{\textit{Apache Hadoop}}\index{Apache!Hadoop} es un software de procesamiento distribuido que permite almacenar 
y procesar grandes cantidades de datos sobre un \textit{cluster} (\autoref{sec:arquitectura_cluster})
 de máquinas, también llamadas nodos del \textit{cluster}.
\textit{Hadoop} es un proyecto \textit{open source} de la \textit{Apache Software Foundation} creado inicialmente por
\textit{Doug Cuttin}, actualmente en desarrollo y mantenido por la comunidad de software libre.\\
El diseño de \textit{Hadoop} está enfocado a procesar los datos en el mismo nodo donde se encuentran,
llevando el código al dato y evitando así el cuello de botella resultante del tráfico de red
al transferir los datos. Este diseño es conocido como \textit{\textbf{data locality}\index{Data locality}}.
\textit{Hadoop} es escalable y tolerante a fallos, tanto en el almacenamiento de los datos como en el
procesamiento de estos. La tolerancia a fallos la gestiona mediante la replicación de los datos
en 3 copias (por defecto, aunque es configurable), cada una en un nodo distinto del cluster. 
De esta manera facilita así las oportunidades de \textit{data locality}, explicado anteriormente.

\noindent \textit{Apache Hadoop} se compone de dos partes fundamentales:
\begin{description}
  \item[HDFS](\textit{Hadoop Distributed File System})\index{Hadoop!HDFS}, es el software encargado de almacenar 
  y distribuir los datos a través de las máquinas del cluster. Es altamente escalable y tolerante a fallos.
  Cuando un archivo es subido al \textit{cluster}, este es dividido en bloques de $128 mb$ y replicado por
  3 sobre los nodos.
  Su arquitectura esta basada en el tipo maestro-exclavo:
  \begin{itemize}
    \item \textit{NameNode}(Maestro) Contiene los metadatos de los archivos.
    \item \textit{NodeManager}(Exclavo) Contine los datos en sí del archivo.
  \end{itemize}

  \item[YARN] (\textit{Yet Another Resource Negotiator})\index{Hadoop!YARN}, es el encargado de gestionar 
  los recursos del \textit{cluster} (memoria y \textit{CPU} principalmente) e incluye \textit{MapReduce v2} 
  como motor de procesamiento. También es escalable y tolerante a fallos y al igual que \textit{HDFS}, 
  esta diseñado basándose en una arquitectura maestro-exclavo:
  \begin{itemize}
    \item \textit{ResourceManager}(maestro), es el encargado de asignar los contenedores (cajas de
    memoria y \textit{CPU}) en los diversos nodos del \textit{cluster} para el desarrollo de las tareas.
    \item \textit{NodeManager}(exclavo), son los encargados de ejecutar propiamente el código.
  \end{itemize}
\end{description}

\noindent \textit{Hadoop} (y más concretamente \textit{YARN}) utiliza por defecto el motor de procesamiento \textit{MapReduce},
que es explicado en más detalle en la sección \nameref{sec:frameworks_procesamiento_paralelo} 
  (\autoref{sec:frameworks_procesamiento_paralelo}). \\
Además \textit{YARN} no se limita solo a \textit{MapReduce} sino que puede ser utilizado como gestor de recursos
del \textit{cluster} para otros motores de procesamiento como \textit{Spark} o \textit{Flink}\index{Flink} por ejemplo.

\begin{figure}[!htpb]
  \centering
  \includegraphics[scale=0.2]{C:/Users/David/Desktop/TFG/TFGLatex/imagenes/hadoop_logo.png}
  \caption[Logo de \textit{Hadoop}]{Logo de \textit{Hadoop}}
  \label{hadoop_logo}
\end{figure}

\clearpage

\section{Arquitectura de un \textit{cluster}}\label{sec:arquitectura_cluster}
Un \textbf{\textit{cluster}} es un conjunto de máquinaa (ordenadores) conectadas entre sí mediante 
una red de tráfico de datos, y que trabajan como si fuesen una sola máquina.
Cada máquina es independiente del resto, si bien necesitan tener un software instalado en cada una de ellas
que permita la comunicación y sincronización entre todas. Además necesitan una serie de elementos
físicos para que dicha comunicación sea posible.\\*
A cada máquina del \textit{cluster} se le denomina \textit{nodo}, y estos están agrupados en 
conjuntos de nodos llamados \textit{racks}.
Un centro de datos contiene uno o más \textit{clusters}, cada \textit{cluster} contiene 
uno o más \textit{racks} y cada \textit{rack} contiene uno o más nodos de máquinas. 
Los nodos de un mismo \textit{rack} se conectan entre si mediante un \textit{switch} 
(\textit{top rack switch}) y cada \textit{rack} se conecta con uno o varios \textit{switch}.

\begin{figure}[h]
  \centering
  \includegraphics[scale=0.5]{C:/Users/David/Desktop/TFG/TFGLatex/imagenes/cluster_topology.jpg}
  \caption[Topología de un \textit{cluster}]{Topología de un \textit{cluster}}
  \label{cluster_topology}
\end{figure}

\clearpage


\section{Topología de un \textit{cluster Hadoop}}
Como se ha mencionado en la \autoref{sec:que_es_apache_hadoop}, \textit{Hadoop} se compone de dos partes 
fundamentales. Cada una de esas partes se compone de subprocesos que se encargan de distintas tareas por lo que
en el diseño de un \textit{cluster Hadoop} se deben elegir máquinas con una configuración de \textit{hardware} 
adecuada a los servicios que se va a desplegar en ella.

En \textit{clusters} destinados a producción, la mejor opción es utilizar \textit{Cloudera Manager} para
su despliegue. El asistente gráfico y todos los servicios que lleva por detrás irán instalados en una
sola máquina. \\
El resto de servicios propios de \textit{Hadoop} pueden ser instalados en una sola máquina (esto se conoce 
como modo pseudodistribuido\index{Pseudodistribuido}) o en varias máquinas (modo distribuido).
Como regla general, se necesitan mínimo dos máquinas \textit{master}, tres máquinas \textit{worker} y una
máquina \textit{gateway} para tener un \textit{cluster} plenamente funcional.
\newline

Respecto a los servicios de \textit{HDFS} y \textit{YARN} hay que tener ciertas consideraciones,
por ejemplo, la máquina designada como \textit{NameNode} necesitará más memoria \textit{RAM} 
debido a que este guarda toda la información de los metadatos de los archivos en memoria. 
También, las máquinas que designemos como trabajadoras será recomendable que tengan buenos recursos 
de \textit{CPU} y memoria así como conexión de red de alta velocidad, de esta manera los trabajos 
que subamos al \textit{cluster} se ejecutarán más rápido.
Las máquinas \textit{worker} será muy recomendable que lleven instalados los servicios de 
\textit{NodeManager} y \textit{DataNode} conjuntamente, si bien no es obligatorio.
Para un conocimiento más profundo acerca de los servicos de \textit{Hadoop}, ver 
\url{http://hadoop.apache.org/docs/current/}
\newline

A continuación, se muestra un esquema de un \textit{cluster Hadoop} con una configuración básica de los dos 
servicios mencionados anteriormente.

\begin{figure}[h]
  \centering
  \includegraphics[scale=0.5]{C:/Users/David/Desktop/TFG/TFGLatex/imagenes/hadoop_topology.jpg}
  \caption[Servicios de un \textit{cluster Hadoop}]{Servicios de un \textit{cluster Hadoop}}
  \label{hadoop_topology}
\end{figure}

\clearpage