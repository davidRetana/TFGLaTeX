
\begin{table}
  \centering \textbf{\textsc{Sobre el autor}} % texto cabecera
  
  \begin{tabular}[t]{crl} % tabla
    \multirow{4}{*}{\includegraphics[scale=0.05]{C:/Users/David/Desktop/TFG/TFGLatex/imagenes/lorenz.png}} & \textsf{Nombre} & David Retana Ribeiro \\
                            & \textsf{Titulación} & Grado en Matemáticas \\
                            & \multirow{2}{*}{\textsf{correos}} & \texttt{davidretanaribeiro@gmail.com} \\
                            &                          & \texttt{dr4293@outlook.com} \\
    \textsf{LinkedIn} & \multicolumn{2}{l}{\url{https://www.linkedin.com/in/david-retana-ribeiro-519a56147/}} \\
    \textsf{GitHub}   & \multicolumn{2}{l}{\url{https://github.com/davidRetana}} \\
    \textsf{Kaggle}   & \multicolumn{2}{l}{\url{https://www.kaggle.com/davidretana}} \\
  \end{tabular}
  \label{sobre_el_autor} % etiqueta
\end{table}

\vspace*{1cm}

\noindent \textbf{Convenciones en la escritura del documento}:
\begin{itemize}
  \item Se usará \textit{letra en cursiva} para designar aquellos términos en inglés que no son 
        traducidos al español, como por ejemplo \textit{machine learning}, \textit{cluster}...\\
        También se usará para designar nombres propios como \textit{Creative Commons} o \textit{Apache}.
  \item La letra en \textbf{negrita} quedará reservada para hacer hincapié en ciertos términos que 
        quieran ser remarcados bien sea porque son importantes para el desarrollo del capítulo 
        o porque en ellos se base la idea a explicar en el capítulo.
  \item Las notas al pie de página se usan para explicar conceptos de manera breve y concisa, 
        así como evitar confusiones en la utilización de términos ambiguos.
  \item En numerosas ocasiones aparecerá ejemplos de códigos fuente y comandos de \textit{shell}, 
        éstos aparecerán destacados en un recuadro. Para los comandos \textit{UNIX}, los argumentos encerrados 
        entre signos de desigualdad (< >) indicarán parámetros a completar por el usuario mientras que 
        si están encerrados por corchetes ([ ]) indica que son opcionales.
  \item El código fuente escrito en \textit{Python} seguirá el estilo marcado por 
        \href{https://www.python.org/dev/peps/pep-0008/}{PEP8}.
  \item En el \autoref{chap:implementacion_paralela}, se utiliza un lenguaje matemático para describir 
        con precisión el modelo de algunos algoritmos que se estudian. Al comienzo de dicho capítulo 
        se explica en detalle la notación utilizada.
  \item Los enlaces a paginas web son marcados en color \textcolor{blue!80!black}{azul}, mientras que 
        las referencias a puntos de este documento están marcadas de color %\textcolor{violet!50!black}{violeta}.
        \textcolor{Brown}{marrón}.
\end{itemize}

\vspace*{0.5cm}

\noindent Se asume que el lector de este documento tiene una base en matemáticas y estadística así como en algún 
lenguaje de programación, especialmente \textit{Python}. También es recomendable que el lector este
familiarizado con entornos \textit{UNIX} y tenga conocimientos básicos del uso de la terminal.
Este documento no esta enfocado a explicar el funcionamiento de los algoritmos de \textit{machine learning} 
de los que se habla, si no que se centra en desarrollar técnicas que permitan programar estos algoritmos 
de manera paralela en entornos distribuidos de computación.
\newline

\vfill
\begin{flushright}
Este documento ha sido escrito en \LaTeX , usando Texmaker $4.5$ \\
(compiled with Qt $5.2.1$ and Poppler $0.26.5$).\\
Las imágenes han sido realizadas con \url{https://www.draw.io/}.\\
El código fuente se encuentra disponible en mi página de \href{https://github.com/davidRetana/TFGLaTeX}{GitHub}.
\end{flushright}

\clearpage
