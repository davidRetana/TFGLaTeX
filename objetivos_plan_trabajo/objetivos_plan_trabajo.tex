
\chapter*{Objetivos y plan de trabajo}\label{objetivos_plan_trabajo}
\markboth{Objetivos y plan de trabajo}{} % para que la cabecera coincida bien con el capítulo
\addcontentsline{toc}{chapter}{Objetivos y plan de trabajo}

%%%%%%%%%%%%%%%% Objetivos %%%%%%%%%%%%%%%%%
Con la realización de este proyecto se pretende conseguir desplegar un \textit{cluster} de máquinas instalando el
\textit{software Hadoop} en ellas y posteriormente instalar el servicio de \textit{Spark} y la librería \textit{mrjob}
de \textit{Python}. Para la realización de este objetivo se usara una herramienta llamada \textit{Cloudera Manager} 
(ver: \autoref{apendix:cloudera}) que nos guiará en el proceso de instalación. \\
En cuanto a la parte de algoritmia, esta consistirá en desarrollar algoritmos paralelos de \textit{machine learning} 
usando el \textit{cluster} construido anteriormente para testar su eficiencia. En total se desarrollarán 4 algoritmos,
dos supervisados y dos no supervisados usando \textit{MapReduce} y \textit{Spark} como \textit{frameworks}.
\newline

\subsubsection*{Objetivos}
\begin{itemize}
  \item Instalación de un \textbf{\textit{cluster Hadoop}} de máquinas virtuales.
  \begin{itemize}
    \item nivel físico: levantar en un servidor las máquinas virtuales con la imagen de \textit{centOS} personalizada.
    \item nivel lógico o de \textit{software}: Instalación de \textit{Apache Hadoop}, \textit{Apache Spark} 
          y \textit{mrjob}.
  \end{itemize}
  \item Desarrollo de algoritmos de \textbf{\textit{machine learning}} de manera distribuida en dicho \textit{cluster}.
  \begin{itemize}
    \item Algoritmos de aprendizaje supervisado.
    \begin{itemize}
      \item Regresión lineal (\textit{Spark}) y \textit{NaiveBayes} (\textit{MapReduce}).
    \end{itemize}
    \item Algoritmos de aprendizaje no supervisado.
    \begin{itemize}
      \item Detección de anomalías (\textit{MapReduce}) y \textit{K-Means} (\textit{Spark}).
    \end{itemize}
  \end{itemize}
\end{itemize}

%%%%%%%%%%%%%% Plan de Trabajo %%%%%%%%%%%%%%%
\subsubsection*{Plan de trabajo}
La instalación de un \textit{cluster} desde 0 es un proceso complejo y que requiere de conocimientos tanto a nivel de 
\textit{hardware} como a nivel de \textit{software} ya que para su realización es necesario una infraestructura física 
y una infraestructura lógica.\\
La infraestructura física se realizará levantando varias máquinas virtuales desde un único servidor y en una red local
de internet. Estas máquinas virtuales simularán máquinas físicas a efectos prácticos ya que cada una posee su propia
dirección \textit{IP}, \textit{CPU}, memoria...
Las imágenes de \textit{centOS} (\textit{Community ENTerprise Operating System}, distribución \textit{gnu/linux}) 
que correrán como sistema operativo se han modificado siguiendo los pasos explicados
en \url{https://github.com/davidRetana/custom_centOS} para que puedan albergar un \textit{cluster}.
\newline

\noindent En la \textbf{\autoref{part:despliegue}} (\nameref{part:despliegue}) el procedimiento será el siguiente:\\
Una vez las máquinas \textit{centOS} estén levantadas y funcionando correctamente en el servidor, comenzaremos la
instalación del \textit{software Hadoop} en cada uno de los nodos mediante conexiones \textit{SSH}\index{SSH} 
(\textit{Secure SHell}, protocolo criptográfico con conexiones cifradas para acceder a servidores remotos.)
, previo reparto de roles entre cada nodo, como se detalla en la \autoref{asignacion_roles_cluster}.
Hecho esto, ya tendríamos un \textit{cluster} donde a continuación instalaremos el servicio de \textit{Spark} y la 
librería \textit{mrjob}. Estos dos programas hacen de \textit{framework} de procesamiento y permiten utilizar
toda la potencia de computo de nuestro \textit{cluster} previamente desplegado.
\newline

\noindent En la \textbf{\autoref{part:analisis_datos}} (\nameref{part:analisis_datos}) el plan consiste en hacer
una pequeña introducción al \textit{machine learning}, para que sirve y por que utilizarlo conjuntamente con el 
\textit{Big Data}. Para la realización de esta sección es necesario el trabajo realizado en la primera 
ya que utiliza el \textit{cluster} desplegado para desarrollar los dos tipos de algoritmos estudiados: 
aprendizaje supervisado y aprendizaje no supervisado. Estos algoritmos se intentarán escribir con una sintaxis 
clara y concisa, y además, se realizarán de manera eficiente dentro de lo posible.


