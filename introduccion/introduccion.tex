\chapter*{Introducción}%\label{capitulo de introducción}
\markboth{Introducción}{} % para que la cabecera coincida bien con el capítulo
\addcontentsline{toc}{chapter}{Introducción}

Vivimos en la era de los datos, cada día se producen mas y mas datos que
necesitan ser almacenados y procesados para poder sacarles beneficio.
En los últimos 10 años se ha generado mas información que el acumulado de años
anteriores y es por esta razón  por la que la manera de almacenar y procesar
los datos ha cambiado.
El \textbf{\textit{Big Data}}\index{Big Data}\label{big_data_def} nace como un concepto para hacer referencia a un 
conjunto de datos masivo, que se origina debido a la incapacidad de los sistemas tradicionales 
de almacenar y procesar toda la información disponible.
La manipulación de grandes cantidades de datos ha de enfrentarse a varios
retos, conocidos como las 3 v's del \textit{Big Data}.
\begin{itemize}
  \item Velocidad (Procesar los datos en un tiempo razonable)
  \item Volumen (Tener la tecnología suficiente para abordar el volumen de datos existente)
  \item Variedad (Saber tratar los distintos tipos de datos en sus diversos formatos)
\end{itemize}
Estos 3 componentes son los que hacen entender el \textit{Big Data} como un 
concepto nuevo. Adicionalmente se incluyen Veracidad y Valor como nuevos conceptos que deben cumplir los datos.
Hemos pasado de hablar en \textit{Gygabytes} o \textit{Terabytes}, a hablar 
en \textit{Petabytes} o incluso \textit{Exabytes}, magnitudes muy por encima de las soportadas 
por las máquinas tradicionales.\\
Para solventar esta carencia, la solución ya no consiste en trabajar contra una sola máquina
sino que consiste en agrupar máquinas y hacerlas trabajar en paralelo simulando que fueran una sola máquina.

% Que me ha llevado a hacer este trabajo y objetivos que quiero conseguir (por encima)
\subsubsection*{Motivación para la realización de este trabajo}
La motivación a la hora de realizar este trabajo ha sido la necesidad de desarrollar algoritmos 
de \textit{machine learning} de manera paralela, ya que los sistemas tradicionales no soportan 
el entrenamiento de estos algoritmos bien sea por falta de memoria o por falta de capacidad de computo.\\
La librería \textit{sklearn} de \textit{Python}, el \textit{software Matlab} o el lenguaje de programación \textit{R}
(orientado al cálculo estadístico) son ejemplos de sistemas para el desarrollo de algoritmos de \textit{machine learning}.
Estos sistemas funcionan muy bien pero solo en un conjunto pequeño de datos ya que si el tamaño del \textit{dataset}
crece, se vuelven incapaces de procesarlo. Esto es así porque dichos sistemas meten los datos en memoria para
procesarlos por lo que la limitación aquí vendría en la \textit{CPU} y en la memoria máxima de la máquina en cuestión 
donde se ejecute.\\
Los algoritmos de \textit{machine learning} se nutren de los datos por lo que cuanto más tengamos más
preciso será nuestro modelo construido con dichos datos.

Por esta razón, los objetivos que pretendo conseguir con este trabajo es la realización de estos 
algoritmos utilizando las distintas herramientas existentes para el manejo de grandes volúmenes de 
datos y el procesado de los mismos, entiéndase \textit{Hadoop}, \textit{MapReduce}, \textit{Spark}, etc.\\
Con estas 3 tecnologías será suficiente para realizar los objetivos marcados y abrir unas líneas futuras 
de investigación para las cuales este trabajo sea de ayuda.
\newline

A nivel personal, la motivación para realizar este trabajo radica en la puesta en práctica de mis conocimientos
matemáticos adquiridos a lo largo de la carrera en Ciencias Matemáticas así como también los conocimientos
aprendidos en el mundo laboral trabajando en temas de \textit{Big data}.

\newpage

% Esquema del trabajo
\subsubsection*{Estructuración del documento}
Este documento se estructura en 3 partes:
\begin{itemize}
  \item \autoref{part:despliegue} (\nameref{part:despliegue})
  \item \autoref{part:analisis_datos} (\nameref{part:analisis_datos})
  \item \autoref{part:apendice} (\nameref{part:apendice})
\end{itemize}
La \textbf{primera parte} consta de una introducción a \textit{Hadoop} y despliegue de un \textit{cluster} con el servicio
complementario de \textit{Spark} y la librería \textit{mrjob} de \textit{Python}.\\
La \textbf{segunda parte} consiste análisis de datos donde se desarrollaran algoritmos de 
\textit{machine learning} de manera distribuida, tanto de aprendizaje supervisado como de aprendizaje no supervisado.\\
La \textbf{tercera y última parte} consta de un apéndice de información útil acerca de sitios web 
donde poner en practica los conocimientos adquiridos mediante competiciones y colaboración con otros equipos de
científicos de datos. Además se habla de \textit{Cloudera} como proveedor de servicios de \textit{Big Data}.
\newline

Cada sección dentro de análisis de datos consta de una primera parte donde se explica la utilidad del algoritmo 
y sus casos de uso, a continuación una segunda parte donde se explica las matemáticas detrás del algoritmo y 
finalmente una ultima parte donde se desarrolla el código de manera distribuida con la inclusión del código fuente.

\vspace*{1.5cm}

\begin{quote}
    'Information is the oil of the $21st$ century, and analytics is the combustion engine'.
	 \newline \raggedleft \textit{Peter Sondergaard}
\end{quote}

\clearpage