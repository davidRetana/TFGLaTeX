\chapter{Kaggle y KDD}
\textbf{Kaggle}\index{Kaggle} es una plataforma que aloja datos de diversas fuentes y organiza 
competiciones para que todo aquel que desee pueda desarrollar sus modelos predictivos y analíticos 
con el objetivo de conseguir el mayor desempeño.
Esta página pone en contacto diversos perfiles de personas (científicos de datos, mineros de datos...) 
con diversos problemas que a menudo exponen compañías y premian a aquellos equipos de personas que 
obtengan la mejor puntuación.
La plataforma tiene una serie de conceptos sobre los que se desarrolla:
\begin{description}
  \item[Competiciones] Permite a las empresas ponerse en contacto con los científicos de datos de la 
  comunidad Kaggle para resolver determinados problemas para su propio beneficio. Los mejores equipos 
  reciben una compensación económica así como puntos para el ranking interno de Kaggle.
  Las competiciones estas clasificadas por nivel de dificultad, variando desde un nivel principiante 
  para iniciarse en el mundo de la ciencia de datos hasta un nivel experto en el cual obliga a los 
  participantes a desarrollar un proyecto de \textit{machine learning} de inicio a final (\textit{end to end}), esto es, 
  preprocesamiento, ingeniería de características, elección del modelo, evaluación...
  \item[Datasets] La plataforma pone a disposición de todo aquel que quiera datos para entrenar de muy 
  diversas fuentes y con diversos objetivos. Nos encontramos con datos que van desde clasificación, 
  regresión, clusterización...
  \item[Kernels\index{Kernels}] Los kernels en Kaggle son \textit{scripts} de código que pueden ser ejecutados 
  en la nube y sirven de ayuda al resto de la comunidad para iniciarse en un cierto problema, preprocesar un 
  conjunto de datos, construir un modelo...
  Los kernels permiten ser valorados por el resto de usuarios que pueden premian tu trabajo con votos y comentarios.
\end{description}
Además de todo lo mencionado anteriormente, Kaggle dispone de un foro para discutir las diversas 
problemáticas que puedan surgir a cada usuario. Conecta a miles de \textit{Data Scientish} de todo el mundo 
para que intercambien ideas y conocimientos.
La plataforma fue fundada por \href{https://en.wikipedia.org/wiki/Anthony_Goldbloom}{Anthony Goldbloom} 
en el año 2010 y se puede acceder a través de \url{https://www.kaggle.com/}
\newline

\vspace*{1cm}

\noindent\textbf{KDD}\index{KDD} viene de sus siglas en inglés \textit{Knowledge Discover Dataset}, y se refiere al hecho 
de sacar información útil de un \textit{dataset}, es decir, extraer conocimiento de los datos.
En su página web \url{http://www.kdd.org/} podemos encontrar toda la información relacionada con lo 
que hacen y a que se dedican. Organizan conferencias y eventos acerca de \textit{KDD}, 
publican \textit{papers}\footnote{documentos que contienen trabajos científicos}
y noticias acerca de temas de innovación, investigaciones y demás relacionadas con el \textit{KDD}.
Cada año lanza una competición abierta al público para que todo aquel interesado pueda descargarse 
el conjunto de datos que proporciona y realizar la tarea que se busca. Una de las \textit{KDD Cup} mas famosas 
fue la del año 1999 \url{http://kdd.ics.uci.edu/databases/kddcup99/kddcup99.html}.
\newline

\vfill

\noindent \textbf{Algunos repositorios de \textit{datasets} de interés:}\\
\url{https://www.kaggle.com/datasets}\\
\url{http://archive.ics.uci.edu/ml/index.php}\\
\url{http://deeplearning.net/datasets/}\\


\chapter{Cloudera}\label{apendix:cloudera}
\textbf{\textit{Cloudera}}\index{Cloudera} es una compañía que proporciona software basado en 
\textit{Apache Hadoop}, formación y soporte técnico. 
De dicho software, en este trabajo se ha usado \textit{Cloudera Manager}, aquí explicamos de manera general 
las 2 posibles opciones para desplegar un \textit{cluster}
\begin{itemize}
  \item \textit{CDH} (\textit{Cloudera Distribution Hadoop})\index{Cloudera!CDH} es una distribución de 
  \textit{Hadoop} modificada por \textit{Cloudera} que permite instalar \textit{Hadoop} con una serie 
  de paquetes para abstraer al programador de tareas como la instalación manual de todos los servicios 
  de un \textit{cluster}, creación de usuarios, mantenimiento...
  
  \item \textit{Cloudera Manager}\index{Cloudera!Manager} (CM) es un asistente gráfico que mediante 
  una API REST\index{Api Rest} permite crear  y gestionar \textit{clusters} de máquinas de una manera sencilla 
  y visual. Con esta herramienta se pueden desplegar servicios en el \textit{cluster}, montar seguridad 
  (\textit{Kerberos, Sentry}...), acceso unificado a los \textit{logs}\index{Archivo log}
  \footnote{Archivos que solo permiten añadir contenido al final del mismo y sirven para saber de manera 
  mas detallada lo que pasa en la ejecución de un programa} y demás opciones.
  También ofrece métricas y estadísticas del \textit{cluster} tales como uso de 
  \textit{CPU}, tráfico de red, I/O de disco...
  Esta opción sería la mas sensata cuando el \textit{cluster} tiene muchos nodos o tiene muchos servicios
  instalados en el, ya que mantenerlo se volvería una tarea bastante tediosa y propensa a fallos.
\end{itemize}

Tanto \textit{CDH} como \textit{CM} son de código abierto y cualquiera puede acceder a ellos 
bajo licencia de \textit{Cloudera}.
En este documento se ha detallado la instalación de un \textit{cluster} usando Cloudera Manager en la
\autoref{sec:instalacion_hdfs_yarn}

\begin{figure}[!htpb]
  \centering
  \includegraphics[width=\textwidth]{C:/Users/David/Desktop/TFG/TFGLatex/imagenes/cloudera_logo.png}
  \caption[Logo de Cloudera]{Logo de Cloudera}
  \label{cloudera_manager}
\end{figure}
